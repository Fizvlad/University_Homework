\documentclass{article}

\usepackage{cmap} % Улучшенный поиск русских слов в полученном pdf-файле
\usepackage[T2A]{fontenc} % Поддержка русских букв
\usepackage[utf8]{inputenc} % Кодировка utf8
\usepackage[english, russian]{babel} % Языки: русский, английский
\usepackage[left = 2cm, right = 2cm, top = 2cm, bottom = 2cm, bindingoffset = 0cm]{geometry} % Отступы

\begin{document}
    \begin{titlepage}
        \newpage
            \begin{center}
                \Large Библиотека для работы с файлами FASTA:\\
                Общая информация
            \end{center}
    \end{titlepage}
    
    \part{FASTA файлы}
        \section{Общая информация о файлах FASTA}
            \paragraph{}
                    \emph{FASTA} - формат файлов, содержащих информацию о некоторой нуклеотидной или полипептидной последовательности.
                    Помимо самой \textbf{последовательности}, FASTA файл хранит её \textbf{идентификатор и описание} в заголовке.
        \section {Структура файла}
            \subsection{Заголовок файла}
                \subsubsection{Пример заголовка FASTA файла:}
                    \begin{quote}
                        >M57671.1 Octodon degus insulin mRNA, complete cds
                    \end{quote}
                    \paragraph{В нашем случае:}
                        \begin{itemize}
                          \item \emph{<<M57671>>} - идентификатор данной последовательности
                          \item \emph{<<.1>>} - версия
                          \item \emph{<<Octodon degus insulin mRNA, complete cds>>} - описание последовательности
                        \end{itemize}
                \subsubsection{Идентификатор}
                    В начале каждого FASTA файла, сразу же за символом '<', находиться его идентификатор (AC, accession).\\
                    Идентификатор позволяет легко найти последовательность в генетических базах (например, GenBank).
                \subsubsection{Версия}
                    После точки в идентификаторе следует версия генетической последовательности. Версию последовательности зачастую не отделяют от идентификатора. Существуют идентификаторы без указания версии.
                \subsubsection{Описание}
                    Краткое описание генетической последовательности. Разделяется с идентификатором пробелом и оканчивается знаком переноса строки.\\
                    \textbf{Опциональный элемент.}
                    \paragraph{В нашем случае:}
                        \begin{itemize}
                          \item \emph{<<Octodon degus>>} - вид животного (грызун Дегу)
                          \item \emph{<<Insulin mRNA>>} - мРНК, кодирующая инсулин
                          \item \emph{<<Complete cds>>} - последовательность описана полностью от старт-кодона до стоп-кодона
                        \end{itemize}
            \subsection{Последовательность}
                \subsubsection{Общая информация}
                    Последовательность являет собой набор символов, каждый из которых имеет своё значение (см. таблицу ниже).
                    \paragraph{Требования формата:}
                        \begin{itemize}
                          \item Разрешены только символы латинского алфавита (в обоих регистрах), символы '*' и '-'
                          \item Допустимы пробелы и переносы строки, их следует игнорировать
                          \item Цифры не допускаются, хотя некоторые базы данных используют их для обозначения позиции
                        \end{itemize}
                        \begin{table}[h]
                            \caption{Условные обозначения}
                            \label{tabular:CharMeanings}
                            \begin{center}
                                \begin{tabular}{|c|c|c|}
                                    \hline
                                    Символ & Значение & Описание \\
                                    \hline
                                    A & A             & Аденин \\
                                    C & C             & Цитозин \\
                                    G & G             & Гуанин \\
                                    T & T             & Тимин \\
                                    U & U             & Урацил \\
                                    R & A, G          & Пурины \\
                                    Y & C, T, U       & Пиримидины \\
                                    K & G, T, U       & Кетоновые основания \\
                                    M & A, C          & Основания с аминогруппами \\
                                    S & C, G          & Сильное взаимодействие \\
                                    W & A, T, U       & Слабое взаимодействие \\
                                    B & не A          & C, G, T или U \\
                                    D & не C          & A, G, T или U \\
                                    H & не G          & A, C, T или U \\
                                    V & не T и не U   & A, C или G \\
                                    N & A, C, G, T, U & Нуклеиновая кислота \\
                                    X & маска         &  \\
                                    - & пропуск неопределенной длины &  \\
                                    \hline
                                \end{tabular}
                            \end{center}
                        \end{table}
\end{document} 