\documentclass{article}
\usepackage{cmap} % Улучшенный поиск русских слов в полученном pdf-файле
\usepackage[T2A]{fontenc} % Поддержка русских букв
\usepackage[utf8]{inputenc} % Кодировка utf8
\usepackage[english, russian]{babel} % Языки: русский, английский
\usepackage[left = 2cm, right = 2cm, top = 2cm, bottom = 2cm, bindingoffset = 0cm]{geometry} % Отступы

\begin{document}
    \part{FASTA файлы}
        \section{Общая информация о файлах FASTA}
            \paragraph{}
                    \emph{FASTA} - формат файлов, содержащих информацию о некоторой нуклеотидной или полипептидной последовательности.
                    Помимо самой \textbf{последовательности} FASTA файл способен хранить её \textbf{идентификатор и описание} в заголовке.
        \section {Структура файла}
            \subsection{Заголовок файла}
                \subsubsection{Пример заголовка FASTA файла:}
                    \begin{quote}
                        >M57671.1 Octodon degus insulin mRNA, complete cds
                    \end{quote}
                    \paragraph{В нашем случае:}
                        \begin{itemize}
                          \item \emph{<<M57671>>} - идентификатор данной последовательности.
                          \item \emph{<<.1>>} - версия
                          \item \emph{<<Octodon degus insulin mRNA, complete cds>>} - описание последовательности
                        \end{itemize}
                \subsubsection{Идентификатор}
                    В начале каждого FASTA файла находиться его идентификатор (AC, accession).\\
                    Идентификатор позволяет легко найти последовательность в генетический базах (например, GenBank)
                \subsubsection{Версия}
                    Версия генетической последовательности
                \subsubsection{Описание}
                    Краткое описание генетической последовательности:
                    \begin{itemize}
                      \item \emph{<<Octodon degus>>} - вид животного (грызун Дегу)
                      \item \emph{<<Insulin mRNA>>} - мРНК, кодирующая инсулин
                      \item \emph{<<Complete cds>>} - последовательность описана полностью от старт-кодона до стоп-кодона
                    \end{itemize}
            \subsection{Последовательность}    
\end{document} 