\documentclass{article}
\usepackage{cmap} % Улучшенный поиск русских слов в полученном pdf-файле
\usepackage[T2A]{fontenc} % Поддержка русских букв
\usepackage[utf8]{inputenc} % Кодировка utf8
\usepackage[english, russian]{babel} % Языки: русский, английский

\renewcommand{\labelenumii}{\arabic{enumi}.\arabic{enumii}.}

\begin{document}
    \begin{titlepage}
        \newpage
            \begin{center}
                \Large Библиотека для работы с файлами FASTA:\\
                Техническое задание
            \end{center}
    \end{titlepage}
    
    \part{Техническое задание}
        \begin{enumerate}
          \item \textbf{Наименование}: \\Библиотека для работы с файлами формата FASTA
          \item \textbf{Назначение}: \\Быстрая и эффективная работа с файлами формата FASTA
          \item \textbf{Основание для разработки}: \\Структура формата FASTA          
          \item \textbf{Функции}:
          \begin{enumerate}
              \item \emph{Чтение из файла}: \\Корректный перевод Id, описания (при наличие) и последовательности в формат для дальнейшей работы
              \item \emph{Запись в файл}: \\Создание FASTA файла с данным Id, описанием и последовательностью
              \item \emph{Базовое выравнивание последовательности}: \\Поиск наибольшей одинаковой подпоследовательности в двух последовательностях
              \item \emph{Частотный анализ}: \\Анализ частоты появления различных символов при чтение из файла и при изменение
          \end{enumerate}
          \item \textbf{Структура}: \\Класс хранящий последовательность(последовательность содержиться в отдельном классе) и информацию о ней
          \item \textbf{Интерфейс}: \\Доступ к основным функциям
        \end{enumerate}

\end{document} 